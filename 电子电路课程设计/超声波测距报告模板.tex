\documentclass[a4paper]{ctexart}
\usepackage[left=2cm, right=2cm, top=2.5cm, bottom=2.2cm]{geometry}

\usepackage{fancyhdr}
\usepackage{graphicx, subfigure}
\usepackage{minted}
\setmonofont[]{Fira Code}
\usepackage[shortlabels]{enumitem}
\setenumerate{itemsep=0pt,partopsep=0pt,parsep=\parskip,topsep=0pt, itemindent=4em, leftmargin=0pt, listparindent=2em, label=(\arabic*)}
\setitemize{itemsep=0pt,partopsep=0pt,parsep=\parskip,topsep=5pt}
\setdescription{itemsep=0pt,partopsep=0pt,parsep=\parskip,topsep=5pt}

\usepackage[sort&compress]{gbt7714}
\usepackage{hyperref}
\hypersetup{unicode, colorlinks=true, linkcolor=black, urlcolor=blue}
\usepackage{booktabs}
\usepackage{multicol}

% LaTeX template made by duskmoon
% stored in THU_EXP
% https://github.com/duskmoon314/THU_EXP
% 
% CC BY-SA 4.0


\begin{document}

\pagestyle{plain}

\ctexset{
    section = {
      format = \zihao{4}\heiti,
      beforeskip = {2ex},
      afterskip = {2ex}
     },
    subsection = {
            format = \zihao{5}\heiti,
            beforeskip = {2ex},
            afterskip = {1ex},
        },
    subsubsection = {
            format = \zihao{5}\songti,
            beforeskip = {2ex},
            afterskip = {1ex},
        },
    appendix = {
            name = {附录}
        }
}
\linespread{1}

\title{\textbf{超声测距系统的设计}}
\author{作者\textsuperscript{1},作者\textsuperscript{2}\\
{\small (1. 班级 学号;2. 班级 学号)}
}
\date{}
\maketitle

\thispagestyle{fancy}
\chead{电子电路课程设计(2)}
\lhead{}
\lfoot{
    \small{实验时间:x月xx日-x月xx日 \\ 实验地点:清华大学中央主楼903室}
}
\renewcommand{\headrulewidth}{1pt}

\setcounter{section}{-1}

\begin{multicols}{2}

    \section{课程任务和实验要求}
    \subsection{课程任务}
    《电子电路课程设计》是一门综合应用模拟电路和数字电路理论进行电子系统设计的课程,要求设计并制作具有较完整功能的小型电子系统,它侧重于电子技术理论知识的灵活运用和设计的创新,因此具有系统性、综合性和探索性。课程任务有:
    \begin{enumerate}
        \item 掌握一般电路系统的设计思路和方法
        \item 培养系统观念和工程观念、解决电路实际问题的能力和探索创新精神
        \item 培养实验研究的总结和表达能力
    \end{enumerate}

    \subsection{实验要求}

    设计并制作一个超声测距系统。

    \begin{enumerate}
        \item 基本要求:

              用示波器显示并测量出接收波与发射波的时延,计算出测量距离。

              测量距离大于1m,显示精度为0.01m,数字显示测量结果,并能动态更新。

              测量距离大于2m,显示方式同上。

        \item 提高要求:

              实现测量距离的稳定显示,即显示不闪烁。
    \end{enumerate}

    \section{实验设计}

    可包含总体方案设计、电路框图设计、电路设计和仿真分析,画出完整的最终电路原理图以及电路中关键元器件的作用说明等。

    完整的电路原理图作为附录,放在文末。

    注意图和表的规范化。
    \subsection{图的规范化}

    正文内容。图的宽度不超过7.5cm。图中量的意义要在正文中加以解释。

    \begin{figure}[H]
        \centering
        \fbox{\rule{0pt}{5cm} \rule{7.5cm}{0pt}}
        \caption{图题}
        \label{fig:one_col}
    \end{figure}

    (\textcolor{blue}{曲线图中横纵坐标的物理量用国际标准符号表示,物理量的符号用斜体字母标注,尽量避免使用中、英文的文字段(单词或缩写字母)来代替符号;单位符号应使用正体字母标注,用“/ ” 与量纲单位隔开(如:V/V;E/a.u.;I/(μA•cm-2);t/℃),刻度线应在框内侧})标值的有效数字为3位。图中文字:宋小5)

    \subsection{表格的规范化}

    表格的设计应该科学、明确、简洁,具有自明性。表格应采用三线表,小表宽度小于7.5 cm,大表宽度为12~15cm 。表身中同一栏各行的数值应以个位(或小数点),且有效位数相同。上下左右相邻栏内的文字或数字相同时,应重复写出。表中“量”的意义要在正文中加以解释

    \begin{table}[H]
        \centering
        \caption{表题}
        \label{tab:one_col}
        \begin{tabular*}{7.5cm}{ccccc}
            \toprule
            \\
            \midrule
            \\
            \\
            \\
            \\
            \bottomrule
        \end{tabular*}
    \end{table}

    \subsubsection{量和单位的书写规则}

    正文、图表中的变量都要用斜体字母,对于矢量和张量使用黑斜体,pH采用正体;

    注意区分量的下标字母的正斜体:凡量符号和代表变动性数字及坐标轴的字母作下标,采用斜体字母。

    单位符号采用正体字母。注意区分单位符号的大小写:一般单位符号为小写体,来源于人名的单位符号首字母大写。

    \section{实验数据整理和分析}

    整理实验数据或示波器存储下来的图片,分析包含实验现象分析、误差分析等

    \section{实验总结}

    总结实验的完成情况、实验过程中遇到的问题以及解决办法、自己有什么收获等。

    \textcolor{blue}{参考文献多作者的需给出前三作者名,采取姓前,名后,“姓”要全称全大写,“名”要缩写。例如期刊的格式为“作者.文题[文献标识].期刊名,年,卷(期):起止页码”,(文献类型标识为:专著M,论文集C,期刊J,学位论文D,报告R,标准S,专利P,工具书K等)。}

    参考文献采用bibtex,使用gbt7714宏包,现默认使用同一目录下的ref.bib文件。例子见下,提交前清注意颜色。

    \nocite{example1,example2,example3,example4,example5,example6,example7}

    \textcolor{blue}{\small{
            \bibliographystyle{gbt7714-numerical}
            \bibliography{ref}
        }}

\end{multicols}

\newpage

\appendix

\section{电路原理图}

电路原理图(包括元件参数)要和实际硬件电路一致且清晰易读。

\section{实验原始数据}

拍照后贴于此。

\section{成本分析}

估算所设计的电路的造价成本:

电阻和电容器统一按每个0.1元计算。

其他元器件的价格可在立创商城(\url{www.szlcsc.com}),得捷电子(\url{https://www.digikey.com.cn/}),淘宝等处查询。

如果你的设计利用了FPGA,其成本按照资源利用率折算,FPGA的芯片价格为350元,如果资源利用率为10\%,则按照350元*10\%=35元来估算成本。

电源和显示部分的成本统一按20元计算。

\end{document}